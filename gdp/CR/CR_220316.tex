\documentclass[12pt,a4paper,final]{report}
\usepackage[utf8]{inputenc}
\usepackage[french]{babel}
\usepackage[T1]{fontenc}
\usepackage{amsmath}
\usepackage{amsfonts}
\usepackage{amssymb}
\usepackage{lmodern}
\usepackage[left=2cm,right=2cm,top=2cm,bottom=2cm]{geometry}

\begin{document}


\begin{center}
\Large \textbf{Compte-rendu de la réunion du 16 mars 2022}
\end{center}

\textbf{Horaire et lieu : 16h35 -- Salle 1.15}
\vskip 0,5cm

\begin{center}
\begin{tabular}{|l|r|}
    \hline
    Présent & Absent \\
    \hline
    Tom BENE & Personne \\
    Alexandre LI & \\
    Camille MOUSSU & \\
    Guillaume RICARD & \\
    \hline
\end{tabular}
\end{center}

\begin{flushleft}
    \textbf{Ordre du jour}
\end{flushleft}

\begin{enumerate}
    \item Première réunion du groupe
	\item Élection du chef de projet
    \item Définition du projet
	\item Début de la répartition des tâches pour la définition du projet

\end{enumerate}

\begin{flushleft}
    \textbf{Élection du Chef de projet}
\end{flushleft}
Camille est candidate, un vote est effectué et Camille est élue avec 3 votes pour et une abstention
\begin{flushleft}
    \textbf{Définition du projet}
\end{flushleft}

\begin{flushleft}
    \textit{Wordle}
\end{flushleft}
Le groupe décide d'adopter différents moyens pour la mise en place du site:
\begin{enumerate}
    \item JavaScript pour le frontend (plus précisément React) et Tailwind pour la mise en page
    \item La partie API en Flask avec SQLalchemy pour les appels à la base de données
    \item La base de donnée en SQlite 
\end{enumerate}
Le groupe discute aussi des fonctionnalités à faire:
\begin{enumerate}
    \item Jeu paramétrable par le joueur (longueur des mots, nombre max d'essais, choix du dictionnaire)
    \item Sauvegarder dans une base de données les parties jouées (pour chaque joueurs et en général)
    \item Mode difficile (on utilise forcément les indices)
    \item Plusieurs dictionnaires (Anglais, Français, Langues régionales)
\end{enumerate}
\newpage 
\begin{flushleft}
    \textit{Solveur}
\end{flushleft}
Le groupe est assez hésitant sur cette partie puisqu'il y n'a pas encore eu de cours sur le C et qu'aucun d'entre nous n'a d'expérience.\\
Le groupe envisage quand même une petite fonctionnalité annexe si le solveur arrive a être mis en place sans trop de difficulté: une version automatique du solveur.


\begin{flushleft}
    \textbf{Répartition des taches:}
\end{flushleft}
\begin{itemize}
    \item Tom: définition du front end
    \item Alexandre: pré-définition du solveur
    \item Camille: API et BD
    \item Guillaume: pré-définition du solveur
\end{itemize}

\begin{flushleft}
    \textbf{Réunion suivante \textit{(Fin de définition)} :} mardi 22/03/2022 à 16h
\end{flushleft}

\textit{Fin de réunion à 17h10}

\end{document}