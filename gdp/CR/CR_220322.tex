\documentclass[12pt,a4paper,final]{report}
\usepackage[utf8]{inputenc}
\usepackage[french]{babel}
\usepackage[T1]{fontenc}
\usepackage{amsmath}
\usepackage{amsfonts}
\usepackage{amssymb}
\usepackage{lmodern}
\usepackage[left=2cm,right=2cm,top=2cm,bottom=2cm]{geometry}

\begin{document}


\begin{center}
\Large \textbf{Compte-rendu de la réunion du 22 mars 2022}
\end{center}

\textbf{Horaire et lieu : 16h15 -- Salle 1.15}
\vskip 0,5cm

\begin{center}
\begin{tabular}{|l|r|}
    \hline
    Présent & Absent \\
    \hline
    Tom BENE & Personne \\
    Alexandre LI & \\
    Camille MOUSSU& \\
    Guillaume RICARD & \\
    \hline
\end{tabular}
\end{center}

\begin{flushleft}
    \textbf{Ordre du jour}
\end{flushleft}

\begin{enumerate}
    \item Progression dans la définition
    \item To-do list pour la prochaine réunion
\end{enumerate}


\begin{flushleft}
    \textbf{Progression}
\end{flushleft}

\textit{Partie wordle } \\ 
Un schéma du site (non fonctionnel) a été effectué en React par Tom. Camille a de son côté listé les différentes fonctions nécéssaires pour l'API. Les dictionnaires sont mieux stockés sur des API externes (Type datamuse) plutôt que dans une base de donnée que l'on aurait créée. Dans le cas d'un wordle utilisant un dictionnaire particulier, il sera simplement donné dans un fichier texte. 
\vskip 0.25cm
\textit{Partie solveur } \\ 
Guillaume et Alexandre ont réalisé un "état de l'art" sur le solveur où ils donnent l'idée générale de son fonctionnement ainsi que différentes stratégies. Un "complément" sur les bases de la théorie de l'information a également été réalisé afin que l'ensemble du groupe soit au point sur ces notions et pour fixer les conventions. 


\begin{flushleft}
    \textbf{To-do list}
\end{flushleft}

\begin{itemize}
    \item Tom : Finaliser le schéma du site et aider Camille avec l'API
    \item Camille : Finir le listing pour l'API
    \item Alexandre : Réfléchir sur les structures de données utilisées par le solveur 
    \item Guillaume : Réfléchir sur les structures de données utilisées par le solveur 
    \item Tous : Remplir le document de conception
\end{itemize}

\begin{flushleft}
    \textbf{Réunion suivante \textit{(Relecture du document de conception)}} : mardi 29/03/2022 à 16h
\end{flushleft}

\begin{flushleft}
    \textit{Fin de réunion à 17h00}
\end{flushleft}

\end{document}