\documentclass[12pt,a4paper,final]{report}
\usepackage[utf8]{inputenc}
\usepackage[french]{babel}
\usepackage[T1]{fontenc}
\usepackage{amsmath}
\usepackage{amsfonts}
\usepackage{amssymb}
\usepackage{lmodern}
\usepackage[left=2cm,right=2cm,top=2cm,bottom=2cm]{geometry}

\begin{document}


\begin{center}
\Large \textbf{Compte-rendu de la réunion du 5 mai 2022}
\end{center}
\textbf{9h00 -- Salle 1.19}
\vskip 0,5cm

\begin{center}
\begin{tabular}{|l|r|}
    \hline
    Présent & Absent \\
    \hline
    Tom BENE & Personne \\
    Alexandre LI & \\
    Camille MOUSSU & \\
    Guillaume RICARD  & \\
    \hline
\end{tabular}
\end{center}

\begin{flushleft}
    \textbf{Ordre du jour}
\end{flushleft}

\begin{enumerate}
	\item Retour sur la soutenance concernant l'application 
	\item Le solveur
\end{enumerate}

\begin{flushleft}
    \textbf{Retour sur la soutenance concernant l'application}
\end{flushleft}

Le groupe est satisfait de la soutenance, et est prêt à se lancer dans la grande aventure qu'est le solveur. 

\begin{flushleft}
    \textbf{Solveur}
\end{flushleft}

\begin{itemize}
    \item \textit{Implementation de la structure de donnée pour le dictionnaire} \\
    Une première solution proposée pour stocker nos dictionnaires en C est de réaliser une table de hash. Cela a été vu en TP ce qui facilitera son implémentation pour le projet.
    \item \textit{Affichage et récupération des entrées utilisateurs} \\
    Cette partie relativement simple et rapide à concevoir sera réalisée par Tom en suivant les consignes du sujet.
    \item \textit{Stratégie} \\
    Pour "s'échauffer" le groupe va réaliser une première version de solveur où l'on propose à chaque fois les 5 mêmes mots au jeu, et selon les résultats, le 6ème mot doit être le bon. 
    Dès que des résultats satisfaisants arriveront sur cette solution. Le groupe travaillera sur un solveur faisant appel à la théorie de l'information.
\end{itemize}



\begin{flushleft}
    \textbf{Répartition des taches:}
\end{flushleft}
\begin{itemize}
    \item Guillaume : Récupération du mot dans le dictionnaire
    \item Alexandre : Définition de la structure de données pour le dictionnaire
    \item Camille : Stratégie en 6 mots
    \item Tous : Réfléchir à la stratégie "théorie de l'information" et commencer à implémenter les fonctions de base (entropie, etc...) 
\end{itemize}

\begin{flushleft}
    \textbf{Réunion suivante :} vendredi 13/05/2022 à 14h
\end{flushleft}

\textit{Fin de réunion à 9h30}

\end{document}

