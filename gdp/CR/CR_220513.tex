\documentclass[12pt,a4paper,final]{report}
\usepackage[utf8]{inputenc}
\usepackage[french]{babel}
\usepackage[T1]{fontenc}
\usepackage{amsmath}
\usepackage{amsfonts}
\usepackage{amssymb}
\usepackage{lmodern}
\usepackage[left=2cm,right=2cm,top=2cm,bottom=2cm]{geometry}

\begin{document}


\begin{center}
\Large \textbf{Compte-rendu de la réunion du 13 mai 2022}
\end{center}
\textbf{14h10 -- Discord}
\vskip 0,5cm

\begin{center}
\begin{tabular}{|l|r|}
    \hline
    Présent & Absent \\
    \hline
    Tom BENE & Personne \\
    Alexandre LI & \\
    Camille MOUSSU & \\
    Guillaume RICARD  & \\
    \hline
\end{tabular}
\end{center}

\begin{flushleft}
    \textbf{Ordre du jour}
\end{flushleft}

\begin{enumerate}
    \item Gestion de projet: Rapport
	\item Avancement sur le solveur
	\item Proposition de stratégies
\end{enumerate}

\begin{flushleft}
    \textbf{Gestion de projet: Rapport}
\end{flushleft}

\begin{itemize}
    \item \textit{Le rapport} \\
    Tout le monde rédigent le rapport lorsque le temps le permet. Les comptes-rendus des réunions sont à mettre au propre.
\end{itemize}

\begin{flushleft}
    \textbf{Avancement sur la structure de données}
\end{flushleft}

\begin{itemize}
    \item \textit{Input / output} \\
    Tom a rencontré des difficultés pour les saisies de l'utilisateur dans le terminal. \\
    \item \textit{Structure de données} \\
    Pour la structure de données, nous avons décidé d'implémenter une table de hash. Ayant déjà une implémentation des listes suite aux travaux pratiques de structures de données, il faudrait juste faire l'implémentation de la table. \\ 
    \item \textit{Stratégie autre que la "méthode classique"} \\
    La stratégie en 6 essais, avance sans trop de difficulté. Pour finir l'implémentation, il faut la structure de données 
    \item \textit{Structure de données} \\
    L'implémentation de la structure de données est quasiment bouclée, mais Alexandre rencontre quelques difficultés.
\end{itemize}


\begin{flushleft}
    \textbf{Répartition des taches:}
\end{flushleft}
\begin{itemize}
    \item Tom: Finir la partie saisie de commande des joueurs \\
    \item Alexandre \& Tous: Boucler la structure de données \\
    \item Camille \& Guillaume: Fonction pour le calcul d'entropie \\
    \item Tous : Rédaction du rapport \\
\end{itemize}

\begin{flushleft}
    \textbf{Réunion suivante :} Jeudi 19/05/2022 à 14h (reporté à Dimanche 22/05/2022 14h)
    \item Raison: Problème organisationnel
\end{flushleft}

\textit{Fin de réunion à 14h40}

\end{document}