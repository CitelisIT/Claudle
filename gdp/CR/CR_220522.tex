\documentclass[12pt,a4paper,final]{report}
\usepackage[utf8]{inputenc}
\usepackage[french]{babel}
\usepackage[T1]{fontenc}
\usepackage{amsmath}
\usepackage{amsfonts}
\usepackage{amssymb}
\usepackage{lmodern}
\usepackage[left=2cm,right=2cm,top=2cm,bottom=2cm]{geometry}

\begin{document}


\begin{center}
\Large \textbf{Compte-rendu de la réunion du 22 mai 2022}
\end{center}
\textbf{14h00 -- Discord}
\vskip 0,5cm

\begin{center}
\begin{tabular}{|l|r|}
    \hline
    Présent & Absent \\
    \hline
    Tom BENE & Personne \\
    Alexandre LI & \\
    Camille MOUSSU & \\
    Guillaume RICARD  & \\
    \hline
\end{tabular}
\end{center}

\begin{flushleft}
    \textbf{Ordre du jour}
\end{flushleft}

\begin{enumerate}
	\item Solveur et théorie de l'information 
	\item Rapport
\end{enumerate}

\begin{flushleft}
    \textbf{Solveur et théorie de l'information }
\end{flushleft}

Le groupe rencontre des difficultés sur certains pan du solveur. Globalement, le système d'interaction fonctionne - il ne manque que la lecture du fichier texte avec la longueur du mot à trouver -, la partie sur l'entropie fonctionne. Les structures de données utiles sont correctement implémentées. 
Il ne reste qu'à réaliser la partie récupération des mots et de choix du mot à jouer. 

\begin{flushleft}
    \textbf{Rapport}
\end{flushleft}
Le rapport est bien avancé en particulier sur la partie introduction et état de l'art du solveur, il faut continuer à le compléter.


\begin{flushleft}
    \textbf{Répartition des taches:}
\end{flushleft}
\begin{itemize}
    \item Tous : Rédaction du rapport
    \item Tous : Finir le solveur
\end{itemize}

\textit{Fin de réunion à 14h30}

\end{document}

